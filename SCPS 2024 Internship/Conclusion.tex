\section{Conclusion} \label{sec:Conclusion}

This report aimed to determine whether the new treatment of convection using \gls{TDC} by $\texttt{MESA}$ is more physically accurate compared to \gls{MLT}.

Using the Lupus server at Keele University, models were successfully run comparing outcomes using these to theories and, after analysis, a number of interesting effects were noticed.

Convective velocities at the centre were found to be constrained differently when using \gls{MLT} compared to \gls{TDC} and velocities calculated from gradient variables. 

Shell mergers were also observed, though those involving He-entertainment varied little between mixing theories, whilst those occurring at $\sim 1\mathrm{yr}$ to core collapse depended primarily on metallicity, save for the introduction of radiative zones in the envelope with \gls{MLT} when the expansion caused by the merger occurred, differing from the more homogeneous envelope when using \gls{TDC}.

Some oscillatory effects were also observed in some models, though this was thought to be mainly a numerical result, only observed when models were well-resolved. 

Overall, deviation between the outcome of \gls{MLT} and \gls{TDC} is small compared to effects from changes in resolution (significantly altering the core and surface conditions) and metallicity.

That said, those differences observed have the potential to alter the outcome of evolution both during early and late stages, therefore warranting further investigation to ascertain whether the use of \gls{TDC} by $\texttt{MESA}$ is more physically accurate overall, thereby indicating whether it should be the preferred option for future 1D models. 

The methodology used should, however, be improved. Updating only a few parameters every model would provide greater confidence to the cause of observed effects, providing more confidence to future findings. 

In future, \texttt{MESA} \gls{TDC} models should be compared to 3D models at late evolutionary stages, known to accurately model true \gls{TDC}, thereby indicating whether the 1D approximation is a realistic one.