\section{Method}\label{sec:Method}
For each simulation, inlist parameters were edited using the $\texttt{Fortran}$ language, updating the default test suite values to emulate those used by \privcomcite{Whitehead} in version $\texttt{r10398}$. Here, difficulties were found with the difference in syntax between the two versions and how the test suite used here utilises multiple inlists (changing parameters at different stellar evolutionary stages). 

Throughout, the approach was to update parameters and find the effect of this using \gls{MLT} and \gls{TDC} (comparing the outcomes). Parameters were grouped into the categories: mixing, metallicity, timestepping and miscellaneous parameters, edited in these groups from the default to either that used by \citeauthor{Whitehead} or the recommended value for a production run within the test suite (if the parameter did not exist in version $\texttt{r10398}$). 

This method, whilst enabling analysis of a range of observable effects would, however, have benefited from changing only a few parameters at a time to increase confidence as to their causes (not possible here due to time constrains).

Meanwhile, a number of values remained constant, notably including the mass ($20M_{\odot}$) and convection theorem (Schwarzchild; see Section \ref{sec: MLT_Background}).

\subsection{Data Analysis}

To analyse the simulations, profiles (taken every 10 models, storing detailed location-dependent data) and history files (providing more generalised information for each model) were processed using a number of $\texttt{Python}$ libraries, namely $\texttt{mesaPlot}$ (for plotting isotopic abundances and Kippenhahn diagrams), $\texttt{mesa\_reader}$ (for reading profile and history files) and $\texttt{mesa\_EDDY}$ (to plot more detailed Kippenhahn diagrams).

Plotting isotopic abundances over mass at different points allowed analysis of shell mergers (affected by the style of convection) whilst \glspl{HRD} gave an understanding of effects experienced on the stellar surface and graphs of $\gls{Tc}$ by $\gls{rhoc}$ showed those at the core (see Section \ref{sec:AdvancedEvolution}).

\subsection{Limitations}

Due to the nature of 1D stellar models, it is often difficult to discern whether an observed effect has physical meaning, is numerical, or a mixture of both. Therefore, a range of other parameters were plotted, such as surface radius over time and temperature over radius, to thereby aid in understanding results at different points during a model. 

This is also the stimulus for running models with small parameter changes, providing an indication of whether the cause of an effect is the change in a physical parameter (e.g. metallicity) or something numerical (e.g. timestepping).

%note "Coarse" = test run resolution (long timestepping), "Fine" = high resolution (short timestepping)
%note "Cox" refers to MLT

%ensure to refer to equation below - calculation of convective velocity used in Figure \ref{fig:MachMass_EarlySiBurn}


%explain all parts of the equation!
%consider migrating the equation to the background section